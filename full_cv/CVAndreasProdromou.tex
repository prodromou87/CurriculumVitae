\documentclass[11pt]{myres} % default is 10 pt
%\usepackage{helvetica} % uses helvetica postscript font (download helvetica.sty)
%\usepackage{newcent}   % uses new century schoolbook postscript font
\setlength{\textheight}{9.5in} % increase text height to fit resume on 1 page
\newsectionwidth{0pt}  % So the text is not indented under section headings

\usepackage{hyperref}
\usepackage{setspace}
\usepackage[usenames,dvipsnames]{color}
\usepackage{xcolor}
\usepackage{natbib}
\usepackage{bibentry}
\usepackage{array}
\usepackage[margin=3cm]{geometry}
\nobibliography*

\definecolor{lightgray}{gray}{0.8}
\newcolumntype{L}{>{\raggedleft}p{0.06\textwidth}}
\newcolumntype{R}{p{0.8\textwidth}}
\newcommand\VRule{\color{lightgray}\vrule width 0.5pt}

\hypersetup {
	colorlinks=true,
	urlcolor=blue
	}

\begin{document}
\onehalfspacing

\name{ANDREAS PRODROMOU} % the \\[12pt] adds a blank line after name

\address{
		  \emph{10545 Ponder Way} \\
		  \emph{San Diego, CA, 92126}}

\address{
		  \href{mailto:prodromou.andreas@gmail.com} 								   {\underline{prodromou.andreas@gmail.com}} \\
		  \emph{tel: (858) 263-5813}}


\begin{resume}

\singlespacing

\section{Research Interests}
	\begin{itemize}
		\item Heterogeneous and heterogeneous-ISA computer architectures
        \item Heterogeneous memory architectures and emerging memory technologies
        \item Applications of Machine Learning algorithms in computer architecture
	\end{itemize}

\section{Education}
\noindent
	{\color{blue}\textbf{University of California, San Diego}, San Diego, CA} \\
	\emph{Ph.D. Student} in Computer Science and Engineering \\
	September 2013 -- Present \\
	Advisor: Dr. Dean Tullsen \\
	GPA: 3.6/4 \\
	\textbf{Powell Fellow (2013-2016)}

	{\color{blue}\textbf{University of Cyprus}, Nicosia, Cyprus} \\
	\emph{Master of Science} in Computer Engineering, June 2013 \\
	Advisor: Dr. Chrysostomos Nicopoulos \\
	Co-Advisor: Dr. Yiannakis Sazeides \\
	GPA: 8.57/10

	{\color{blue}\textbf{University of Cyprus}, Nicosia, Cyprus} \\
	\emph{Bachelor of Science} in Computer Engineering, June 2011 \\
	Advisor: Dr. Chrysostomos Nicopoulos \\
	GPA: 7.81/10

\section{Research Experience}

{\color{blue}\textbf{Graduate Student Researcher,} June 2013 -- Present} \\
\emph{Department of Computer Science and Engineering, UCSD}, San Diego, CA \\
\begin{itemize} \itemsep -2pt
    \item \textbf{Dynamic memory management for Heterogeneous Memory Architectures.}
    \begin{itemize}
        \item Exploited a ``Majority Element Algorithm'', originally proposed for big data analytics, to achieve greater prediction accuracy on future memory accesses.
        \item Combined MEA algorithm with a ``Divide-and-Conquer'' hardware mechanism for more efficient and scalable memory management than state-of-the-art proposals.
        \item Work published in HPCA'17 conference.
    \end{itemize}
    \item \textbf{Machine-Learning-based scheduling for heterogeneous(-ISA) architectures }
    \begin{itemize}
        \item Trained and evaluated thousands of different ML models over large collection of data collected via simulations (43200 cycle-accurate simulations - 72 workloads on 600 different cores). These models were trained to predict dynamic performance given an application and core characterization.
        \item Developed techniques to reverse-engineer trained ML models and understand how important each input feature is towards the prediction goal. Insights were later used to minimize amount of on-chip counters necessary and overall reduce overheads.
        \item Proposed a ``scheduler efficiency'' metric to bridge the gap between ML metrics and system performance.
        \item Proposed a ``system scheduling difficulty'' metric that quantifies how hard scheduling is for a given collection of cores in a processor. We find that scheduling difficulty is a reasonable architectural trade-off.
    \end{itemize}
\end{itemize}

{\color{blue}\textbf{Co-op Intern,} June 2018 -- Sept. 2018} \\
\emph{NVidia}, Santa Clara, California \\
\begin{itemize} \itemsep -2pt
    \item Competition analysis and projections team
    \begin{itemize}
      \item In-depth characterization of the training phase of various neural network architectures. Compared NVidia's products (GPUs) against competitora products (othe GPUs, as well as various accelerators such as Google's TPU.)
    \end{itemize}
\end{itemize}

{\color{blue}\textbf{Co-op Intern,} June 2016 -- Sept. 2016} \\
\emph{Advanced Micro Devices (AMD)}, Austin, Texas \\
\begin{itemize} \itemsep -2pt
    \item Implementation of a statistical memory simulator.
    \begin{itemize}
    	\item Implemented a tool capable of extrapolating the behavior of large memories after simulating tiny memories.
    \end{itemize}
\end{itemize}

{\color{blue}\textbf{Co-op Intern,} June 2015 -- Sept. 2015} \\
\emph{Advanced Micro Devices (AMD)}, Austin, Texas \\
\begin{itemize} \itemsep -2pt
    \item Research in dynamic memory management in hybrid configurations.
    \begin{itemize}
    	\item Developed dynamic memory management mechanism focusing mainly on scalability to very large memory capacities.
    	\item Utilizing a ``Majority Element Algorithm'' heuristic for extremely efficient activity tracking
    \end{itemize}
\end{itemize}

{\color{blue}\textbf{Special Scientist,} June 2011 -- June 2013} \\
\emph{Department of Electrical and Computer Engineering, University of Cyprus}, Nicosia, Cyprus \\
$\Xi$ Lab (https://www2.cs.ucy.ac.cy/carch/xi/) \\
\begin{itemize} \itemsep -2pt
	\item Part of Eurocloud FP7 Project (http://www.eurocloudserver.com/) -- European research and development program for building a 3D server-on-chip concept integrating low power cores.
	\item Research in Network-on-Chip Reliability.
	\item Designed and evaluated circuit-level modules to detect hardware faults. (Published in MICRO'45 conference)
\end{itemize}

{\color{blue}\textbf{Special Scientist,} June 2010 -- August 2010} \\
\emph{Department of Electrical and Computer Engineering, University of Cyprus}, Nicosia, Cyprus \\
multiCAL (multicore Computer Architecture Laboratory) (\url{www.multical.ece.ucy.ac.cy}) \\
\begin{itemize} \itemsep -2pt
	\item Implementation of a parameterized cycle-accurate Network-on-Chip simulator as part of a Full-System simulator (extensive programming and simulations)
	\item Later awarded as Best Senior Design Project
\end{itemize}

{\color{blue}\textbf{Undergraduate Research Intern,} June 2009 -- August 2009} \\
\emph{Department of Electrical and Computer Engineering, University of Cyprus}, Nicosia, Cyprus \\
KIOS Research Center (\url{www.kios.ucy.ac.cy}) \\
\begin{itemize} \itemsep -2pt
	\item Studied the infrastructure of a Network-on-Chip module
	\item Implemented and assessed (simulations) routing algorithms for such networks
\end{itemize}

\section{Skills}
{\color{blue}Operating Systems}\\
Extensive knowledge of Windows, Linux and Mac OS

{\color{blue}Programming/Scripting Languages}\\
Excellent knowledge of C, C++\\
Excellent knowledge of scripting languages (Perl, Python and Bash)\\
Good knowledge of MATLAB programming environment\\
Familiarized with Verilog HDL, Javascript and ActionScript

{\color{blue}CAD Software}\\
Synopsis Design Compiler\\
Autodesk AutoCAD

\section{Languages}
\begin{tabular}{ll}
\color{blue}Greek & Native speaker \\[5pt]
\color{blue}English & Verbal and written fluency at an advanced level \\
\end{tabular}

\section{Professional Activities}

\textbf{Military Service}, \emph{Greek Cypriot National Guard}, June 2005 -- July 2007\\
Reserve Officer in the Telecommunications Division of the Cyprus Military\\
\begin{itemize}
	\item Attended a military academy for reserve officers in Athens, Greece, specializing in military telecommunications, June 2005 -- Sept 2005
	\item Served a 25-month military service as an officer
	\item Currently a reserve officer ranked as Second Lieutenant
\end{itemize}

-- Webchair for the Design For Reliability (DFR) workshop, held in conjunction with \emph{The 8th International Conference on High Performance and Embedded Architectures and Compiles} (HiPEAC), 2013.

-- Assisted in reviewing manuscripts for conferences such as:
\begin{itemize}
	\item ACM Computing Surveys Journal, 2012
	\item DATE 2013
	\item MICRO 2011, 2012
\end{itemize}

-- IEEE Student Member 2010 -- Present

\section{Recognition}
\begin{tabular}{L!{\VRule}R}
2013&Powell Fellowship for academic years 2013--2016, UCSD Computer Science and Engineering.\\[5pt]
2011&Best Senior Design Project award in the Department of Electrical and Computer Engineering, University of Cyprus for academic year 2010-2011.\\[5pt]
2008&Award for excellent performance from the Department of Electrical and Computer Engineering of University of Cyprus for the academic year 2007-2008.
\end{tabular}

\newpage

\section{Publications}

\begin{tabular}{L!{\VRule}R}
\vspace{1pt}2018&\bibentry{MG_HPCA_2018}\\[5pt]
\vspace{1pt}2017&\bibentry{prodromou_mempod_2017}\\[5pt]
\vspace{1pt}&\bibentry{JHR_SLIC_2017}\\[5pt]
\vspace{1pt}2016&\bibentry{Chrysanthou_nocalert_2016}\\[5pt]
\vspace{1pt}2012&\bibentry{prodromou_nocalert_2012}\\[5pt]
\vspace{1pt}2011&\bibentry{milojevic2012}\\
\end{tabular}

\section{Patents}
\begin{tabular}{L!{\VRule}R}
\vspace{1pt}2016&\bibentry{Prodromou:2016:MemPod}\\
\end{tabular}

\newpage

\section{Theses}
\begin{tabular}{L!{\VRule}R}
\vspace{1pt}2013&\bibentry{prodromou_mscThesis_2013}\\[5pt]
\vspace{1pt}2011&\bibentry{prodromou_bscThesis_2011}\\
\end{tabular}

\nocite*
\bibliographystyle{plain}
\nobibliography{../publications}

\end{resume}
\end{document}











